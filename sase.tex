\documentclass[12pt,a4paper]{article}
\usepackage[utf8]{inputenc}
\usepackage{amsmath,amssymb,amsfonts}
\usepackage{graphicx}
\usepackage{hyperref}
\usepackage{physics}
\usepackage{bm}
\usepackage{mathtools}
\usepackage[margin=1in]{geometry}

\title{Differentiable modeling of x-ray scattering}
\date{\today}

\begin{document}

\maketitle

\begin{abstract}
We examine how the spectral properties of Self-Amplified Spontaneous Emission (SASE) pulses at X-ray Free-Electron Laser (XFEL) facilities affect measured scattering patterns, focusing on shot-to-shot variability and finite bandwidth. A hierarchy of models is presented, from a monochromatic approximation to approximate spectral modeling using differentiable 3D Gaussians and the reparameterization trick.
\end{abstract}

\section{Introduction}
In X-ray scattering experiments at XFEL facilities, the measured signal depends on the SASE pulse characteristics. These pulses exhibit shot-to-shot fluctuations in central energy, spectral width, and overall shape. Depending on resolution requirements and the relative size of broadening contributions from other factors, such as mosaicity, different approximations might be used.

\section{Physical Framework}

\subsection{Scattering Model}
Let $I(\mathbf{q})$ be the total scattering intensity as a function of the momentum transfer vector $\mathbf{q}$. It comprises Bragg (lattice) and diffuse components:
\begin{equation}
I(\mathbf{q}) = I_B(\mathbf{q}) + I_d(\mathbf{q}).
\end{equation}

\subsection{SASE Pulse Properties}
SASE pulses feature a spectral intensity distribution $I_{\text{SASE}}(\omega)$, subject to:
\begin{itemize}
    \item Shot-to-shot fluctuations in central energy $\omega_0$ and spectral width $\sigma_\omega$
\end{itemize}

\subsection{Measurement Process}
The intensity at pixel $(x,y)$ is
\begin{equation}
I_{\text{det}}(x,y) = \int \bigl[I_B(\mathbf{q}(x,y,\omega)) \cdot G_B(x,y,\omega) + I_d(\mathbf{q}(x,y,\omega)) \cdot G_d(x,y,\omega)\bigr]\; I_{\text{SASE}}(\omega)\, d\omega,
\end{equation}
where $\mathbf{q}(x,y,\omega)$ maps pixel position and photon energy to momentum transfer, while $G_B(x,y,\omega)$ and $G_d(x,y,\omega)$ represent the distinct geometric, partiality and polarization factors for Bragg and diffuse components, respectively.

\section{Modeling Approaches by Complexity}

\subsection{Case 1: Stable Monochromatic Approximation}
\textbf{Assumptions:}
\begin{itemize}
\item Negligible shot-to-shot jitter
\item Negligible bandwidth
\end{itemize}
\textbf{Model:}
\begin{equation}
I_{\text{SASE}}(\omega) \approx \delta(\omega - \omega_0),
\end{equation}
\begin{equation}
I_{\text{det}}(x,y) \approx I_B(\mathbf{q}(x,y,\omega_0)) \cdot G_B(x,y,\omega_0) + I_d(\mathbf{q}(x,y,\omega_0)) \cdot G_d(x,y,\omega_0).
\end{equation}

\subsection{Case 2: Shot-to-Shot Jitter, Monochromatic per Shot}
\textbf{Assumptions:}
\begin{itemize}
\item Energy jitter included
\item Each pulse treated as monochromatic
\item Bandwidth still small compared to other effects
\end{itemize}
\textbf{Model:}
For shot $j$,
\begin{equation}
I_{\text{SASE},j}(\omega) \approx \delta(\omega - \omega_j),
\end{equation}
\begin{equation}
I_{\text{det},j}(x,y) \approx I_B(\mathbf{q}(x,y,\omega_j)) \cdot G_B(x,y,\omega_j) + I_d(\mathbf{q}(x,y,\omega_j)) \cdot G_d(x,y,\omega_j).
\end{equation}
An approximate fractional change in intensity due to jitter:
\begin{equation}
\frac{\Delta I_{\text{det}}}{I_{\text{det}}} \approx \left|\nabla_{\mathbf{q}} \ln I \cdot \frac{\mathbf{q}}{|\mathbf{q}|}\right|\frac{\Delta E}{E}\,|\mathbf{q}|.
\end{equation}

\subsubsection*{Question}
How sharp are the features in the diffuse map? Is this fractional change negligible or not, for the diffuse scattering component?

\subsection{Case 3: Shot-to-Shot Jitter with Gaussian Bandwidth}
\textbf{Assumptions:}
\begin{itemize}
\item Energy jitter and bandwidth both matter
\item SASE spectrum approximated as Gaussian
\end{itemize}
\textbf{Model:}
\begin{equation}
I_{\text{SASE},j}(\omega) \approx \exp\!\Bigl[-\frac{(\omega - \omega_j)^2}{2\sigma_{\omega,j}^2}\Bigr],
\end{equation}
\begin{equation}
I_{\text{det},j}(x,y) \approx \int \bigl[I_B(\mathbf{q}(x,y,\omega)) \cdot G_B(x,y,\omega) + I_d(\mathbf{q}(x,y,\omega)) \cdot G_d(x,y,\omega)\bigr]\,
\exp\!\Bigl[-\tfrac{(\omega - \omega_j)^2}{2\sigma_{\omega,j}^2}\Bigr]\, d\omega.
\end{equation}
In reciprocal space, this corresponds to integrating over a "slab" of thickness proportional to $\Delta E/E \times |\mathbf{q}|$.

\subsubsection*{Differentiable Modeling (Monte Carlo)}
Represent the SASE spectrum in $\mathbf{q}$-space as a 3D Gaussian, with covariance $\Sigma_j = L_j L_j^{T}$. The reparameterization trick samples
\[
\mathbf{q} = \mathbf{q}_j + L_j\,\boldsymbol{\epsilon}, \quad \boldsymbol{\epsilon} \sim \mathcal{N}(0,I),
\]
yielding a Monte Carlo estimate of $I_{\text{det},j}(x,y)$. The distinct geometric factors must be applied to Bragg and diffuse components at each sampled point.

\subsection{Case 4: Full SASE Spectral Shape}
\textbf{Assumptions:}
\begin{itemize}
\item General shot-to-shot jitter
\item No simple analytical form for the SASE spectrum
\end{itemize}
\textbf{Model:}
\begin{equation}
I_{\text{det},j}(x,y) = \int \bigl[I_B(\mathbf{q}(x,y,\omega)) \cdot G_B(x,y,\omega) + I_d(\mathbf{q}(x,y,\omega)) \cdot G_d(x,y,\omega)\bigr]\,
I_{\text{SASE},j}(\omega)\,d\omega.
\end{equation}

\subsubsection*{Implementation}
Measure or reconstruct each pulse's spectral profile per shot. Discretize and integrate numerically.

\subsubsection*{Tradeoffs}
Significantly more computationally expensive than case 4.

\section{Practical Implications}

\section{PyTorch Example (Case 3)}
\begin{verbatim}
def scattering_model(q_0, L, I_B_func, I_d_func, G_B_func, G_d_func, num_samples=100):
    # q_0: central q vector (or array) for a pixel
    # L: Cholesky factor of covariance matrix
    # I_B_func: function computing Bragg intensity at q
    # I_d_func: function computing diffuse intensity at q
    # G_B_func: function computing geometric/polarization factors for Bragg
    # G_d_func: function computing geometric/polarization factors for diffuse
    
    # Sample from standard normal
    epsilon = torch.randn(num_samples, 3)
    
    # Reparameterization
    q_samples = q_0.unsqueeze(0) + torch.matmul(epsilon, L.T)
    
    # Compute Bragg + diffuse at each sampled q with appropriate geometric factors
    I_samples = torch.stack([
        I_B_func(q) * G_B_func(q) + I_d_func(q) * G_d_func(q) 
        for q in q_samples
    ])
    
    # Monte Carlo average
    return I_samples.mean()
\end{verbatim}

\section{Conclusion}
A hierarchy of models provides a systematic way to incorporate SASE pulse effects in X-ray scattering experiments. From the simplest monochromatic approximation to full spectral modeling, one can balance computational effort against experimental fidelity. 

\end{document}
