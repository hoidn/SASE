\documentclass[12pt,a4paper]{article}
\usepackage[utf8]{inputenc}
\usepackage{amsmath,amssymb,amsfonts}
\usepackage{graphicx}
\usepackage{hyperref}
\usepackage{physics}
\usepackage{bm}
\usepackage{mathtools}
\usepackage[margin=1in]{geometry}

\title{Effect of SASE Pulse Properties on Diffuse X-ray Scattering Measurements}
\author{Author}
\date{\today}

\begin{document}

\maketitle

\begin{abstract}
In diffuse X-ray scattering experiments at X-ray Free-Electron Laser (XFEL) facilities, the measured scattering signal depends critically on the properties of the Self-Amplified Spontaneous Emission (SASE) pulses used for the measurements. This document examines how the spectral properties of SASE pulses affect the diffuse scattering patterns recorded by a pixel detector, with particular attention to shot-to-shot variability and finite bandwidth effects. We also propose a differentiable approach using 3D Gaussians with the reparameterization trick to model these effects in analysis workflows.
\end{abstract}

\section{Introduction}

In diffuse X-ray scattering experiments at X-ray Free-Electron Laser (XFEL) facilities, the measured scattering signal depends critically on the properties of the Self-Amplified Spontaneous Emission (SASE) pulses used for the measurements. This document examines how the spectral properties of SASE pulses affect the diffuse scattering patterns recorded by a pixel detector.

\section{Physical Framework}

\subsection{Diffuse Scattering Model}

The diffuse scattering intensity as a function of momentum transfer vector $\mathbf{q}$ can be represented as $I_d(\mathbf{q})$. This is the physical quantity we aim to measure, which represents correlations in electron density fluctuations within the sample.

\subsection{SASE Pulse Properties}

SASE pulses are characterized by:
\begin{itemize}
\item A spectral intensity distribution $I_{\text{SASE}}(\omega)$ or equivalently $I_{\text{SASE}}(q)$
\item Shot-to-shot variations in both central energy $\omega_0$ and spectral width $\sigma_\omega$
\item Finite bandwidth even within a single shot
\end{itemize}

\subsection{Measurement Process}

The intensity detected at pixel position $(x,y)$ on the detector can be expressed as:

\begin{equation}
I_{\text{det}}(x,y) = \int I_d(\mathbf{q}(x,y,\omega)) \cdot I_{\text{SASE}}(\omega) \cdot G(x,y,\omega) \, d\omega
\end{equation}

where:
\begin{itemize}
\item $\mathbf{q}(x,y,\omega)$ maps the pixel position and photon energy to momentum transfer
\item $G(x,y,\omega)$ represents geometric correction factors
\end{itemize}

\section{Key Considerations}

\subsection{Shot-to-Shot Variability (Jitter)}

The SASE central energy typically fluctuates by $\Delta E/E \approx 10^{-3}$ to $10^{-2}$ at current XFEL facilities. This means that for a 10 keV X-ray beam, the central energy may jitter by 10-100 eV between shots.

The impact on measured intensity can be estimated by:

\begin{equation}
\frac{\Delta I_{\text{det}}}{I_{\text{det}}} \approx \left|\frac{\partial \ln I_d}{\partial q}\right| \cdot \frac{\Delta E}{E} \cdot |\mathbf{q}|
\end{equation}

For typical diffuse scattering with significant features where $\left|\frac{\partial \ln I_d}{\partial q}\right| \approx 1-10$ \AA{} and $|\mathbf{q}| \approx 0.1-1$ \AA{}$^{-1}$, this yields:

\begin{equation}
\frac{\Delta I_{\text{det}}}{I_{\text{det}}} \approx 10^{-4}-10^{-1}
\end{equation}

\textbf{Conclusion on jitter}: The shot-to-shot jitter can indeed introduce significant stochasticity in the measured intensity, especially at higher resolution (larger $|\mathbf{q}|$) and for diffuse features with sharp gradients. This effect becomes particularly relevant when analyzing small differences in diffuse scattering patterns.

\subsection{Finite Bandwidth of Single Pulses}

SASE pulses typically have a relative bandwidth of $\Delta E/E \approx 10^{-3}$ for a single shot. This leads to an integration over a range of $\mathbf{q}$ values for each pixel:

\begin{equation}
\delta q \approx \frac{\Delta E}{E} \cdot |\mathbf{q}|
\end{equation}

For $|\mathbf{q}| \approx 1$ \AA{}$^{-1}$, this gives $\delta q \approx 10^{-3}$ \AA{}$^{-1}$.

\textbf{Conclusion on bandwidth}: The finite bandwidth means that each detector pixel measures an integral over a small region in reciprocal space rather than a point. However, this effect is generally small compared to other experimental factors unless measuring extremely sharp features in reciprocal space.

\subsection{3D Visualization}

The finite bandwidth effect can be visualized as measuring a thin ``slab" through reciprocal space rather than an exact 2D Ewald sphere. The thickness of this slab is:

\begin{equation}
\text{Slab thickness} \approx \frac{\Delta E}{E} \cdot |\mathbf{q}|
\end{equation}

For typical SASE parameters and $|\mathbf{q}| \approx 1$ \AA{}$^{-1}$, this gives a thickness on the order of $10^{-3}$ \AA{}$^{-1}$.

\section{Practical Implications}

\begin{enumerate}
\item \textbf{Data Processing Strategy}: Multiple shots should be normalized by their spectral distribution and intensity before averaging to reduce the impact of jitter.

\item \textbf{Feature Resolution}: Features in diffuse scattering with characteristic length scales smaller than the slab thickness will be averaged out.

\item \textbf{Simulation Comparison}: When comparing experimental results to theoretical predictions, the experimental data should be understood as representing a weighted average over a small volume in reciprocal space.

\item \textbf{Monochromator Consideration}: Using a monochromator would reduce these effects but at significant cost to intensity.
\end{enumerate}

\section{Differentiable Modeling of Slab Thickness with 3D Gaussians}

If the slab thickness effect proves significant enough to require explicit modeling, a differentiable approach using 3D Gaussians can be implemented. This approach is particularly valuable when fitting theoretical models to experimental data using gradient-based optimization.

\subsection{Gaussian Representation of SASE Spectral Distribution}

The SASE pulse's spectral distribution can be modeled as a 3D Gaussian in reciprocal space:

\begin{equation}
I_{\text{SASE}}(\mathbf{q}) \approx \exp\left(-\frac{1}{2}(\mathbf{q} - \mathbf{q}_0)^T \Sigma^{-1} (\mathbf{q} - \mathbf{q}_0)\right)
\end{equation}

where:
\begin{itemize}
\item $\mathbf{q}_0$ is the center of the distribution (determined by the central energy)
\item $\Sigma$ is the covariance matrix that defines the shape and orientation of the slab
\end{itemize}

The covariance matrix $\Sigma$ can be parametrized to reflect the anisotropic nature of the slab (thin in one direction, extended along the Ewald sphere).

\subsection{Reparameterization Trick for Differentiability}

To make this model differentiable for gradient-based optimization, we can use the reparameterization trick from variational inference:

\begin{equation}
\mathbf{q} = \mathbf{q}_0 + L\mathbf{\epsilon}
\end{equation}

where:
\begin{itemize}
\item $L$ is the Cholesky decomposition of $\Sigma$ (i.e., $\Sigma = LL^T$)
\item $\mathbf{\epsilon}$ is a standard normal random variable
\end{itemize}

This allows gradients to flow through the parameters $\mathbf{q}_0$ and $L$ during optimization.

\subsection{Implementation in Diffuse Scattering Model}

The measured intensity at pixel $(x,y)$ can then be approximated as:

\begin{equation}
I_{\text{det}}(x,y) \approx \frac{1}{N} \sum_{i=1}^{N} I_d(\mathbf{q}_0(x,y) + L\mathbf{\epsilon}_i) \cdot G(x,y)
\end{equation}

where $N$ is the number of Monte Carlo samples used to estimate the integral. This Monte Carlo integration is differentiable with respect to all parameters.

\subsection{Advantages of This Approach}

\begin{enumerate}
\item \textbf{Differentiability}: Enables the use of efficient gradient-based optimization algorithms.

\item \textbf{Flexibility}: The covariance matrix $\Sigma$ can be adapted to model different SASE beam properties and geometries.

\item \textbf{Computational Efficiency}: The reparameterization approach allows backpropagation through the model while maintaining computational tractability.

\item \textbf{Uncertainty Quantification}: Natural extension to modeling shot-to-shot variations by treating the parameters as random variables with priors.
\end{enumerate}

\subsection{Practical Implementation}

In a PyTorch or JAX implementation, this might look like:

\begin{verbatim}
def diffuse_model(q_0, L, I_d_func, num_samples=100):
    # q_0: central q vector for a pixel
    # L: Cholesky factor of covariance matrix
    # I_d_func: function that computes diffuse intensity at a given q
    
    # Sample from standard normal
    epsilon = torch.randn(num_samples, 3)
    
    # Apply reparameterization trick
    q_samples = q_0.unsqueeze(0) + torch.matmul(epsilon, L.T)
    
    # Compute intensity at each sampled q point
    I_samples = torch.stack([I_d_func(q) for q in q_samples])
    
    # Average the intensities (Monte Carlo integration)
    return I_samples.mean()
\end{verbatim}

This framework allows for efficient, differentiable modeling of the finite bandwidth effects in SASE-based diffuse scattering experiments.

\end{document}
